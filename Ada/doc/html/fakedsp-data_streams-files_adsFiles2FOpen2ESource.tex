% Document: ./src/fakelib/fakedsp-data_streams-files.ads
% Source: ./src/fakelib/fakedsp-data_streams-files.ads
% Generated with ROBODoc Version 4.99.43 (Apr  5 2019)
\documentclass{article}
\usepackage{makeidx}
\usepackage{graphicx}
\oddsidemargin 0.25 in 
\evensidemargin 0.25 in
\marginparwidth 0.75 in
\textwidth 5.875 in
\setlength{\parindent}{0in}
\setlength{\parskip}{.08in}

\pagestyle{headings}
\title{FakeDSP API}
\author{Generated with ROBODoc Version 4.99.43 (Apr  5 2019)
}
\makeindex
\begin{document}
\maketitle
\printindex
\tableofcontents
\newpage

<a class="menuitem" href="./fakedsp-data_streams-files_adsData5FStreams2FFiles.tex#robo10">Files</a><a class="menuitem" href="./robo_functions.tex#robo_top_of_doc">Functions</a>\subsection{Files/Open.Source}
\index{unsorted!Open.Source}\index{Functions!Open.Source}
\textbf{SOURCE:}\hspace{0.08in}\begin{verbatim}
   function Open (Filename : String;
                  Format   : File_Type := Unknown)
                  return Data_Source_Access;
\end{verbatim}
\textbf{DESCRIPTION:}\hspace{0.08in}
   Open a Data\_Source (\ref{ch:robo9}) based on the file with the specified filename and
   format (WAV, text-based, ...).   If the special
   filename "-" is used, the standard input is used.



   If Format = Unknown, then:

\begin{itemize}
  \item    if Filename = "-", format Text\_File is used by default, otherwise
  \item    the format is guessed on the basis of the extension (maybe
     in the future we will check the content too).
\end{itemize}


   Sampling frequency and number of channels are Read (\ref{ch:robo6}) from the
   specified file or from the options Read (\ref{ch:robo6}) from the filename


\end{document}
