% Document: ./src/fakelib/fakedsp-card.ads
% Source: ./src/fakelib/fakedsp-card.ads
% Generated with ROBODoc Version 4.99.43 (Apr  5 2019)
\documentclass{article}
\usepackage{makeidx}
\usepackage{graphicx}
\oddsidemargin 0.25 in 
\evensidemargin 0.25 in
\marginparwidth 0.75 in
\textwidth 5.875 in
\setlength{\parindent}{0in}
\setlength{\parskip}{.08in}

\pagestyle{headings}
\title{FakeDSP API}
\author{Generated with ROBODoc Version 4.99.43 (Apr  5 2019)
}
\makeindex
\begin{document}
\maketitle
\printindex
\tableofcontents
\newpage

<a class="menuitem" href="./rootdoc_docRoot2FFakedsp.tex#robo28">Fakedsp</a><a class="menuitem" href="./robo_packages.tex#robo_top_of_doc">Packages</a>\subsection{Fakedsp/Fakedsp.card}
\index{unsorted!Fakedsp.card}\index{Packages!Fakedsp.card}
\textbf{DESCRIPTION:}\hspace{0.08in}
  This package provides the interface to a "virtual DSP card" with an ADC
  and a DAC. The virtual ADC reads sample at a specified sampling frequency
  and at the same instants the virtual DAC updates its output. 
  The user needs to register a "virtual interrupt handler"
  that will be called every time the virtual ADC reads a new sample.
  The interrupt handler can Read (\ref{ch:robo6}) data from the ADC using the function
  Read\_ADC (\ref{ch:robo15}) and can Write (\ref{ch:robo3}) data to the DAC usign Write\_DAC (\ref{ch:robo19}).

  

  The interrupt handler must be a variable of a type that implements 
  (is a descendant of) the interface Callback\_Handler (\ref{ch:robo12}).  Every
  implementation of Callback\_Handler (\ref{ch:robo12}) must provide a procedure Sample\_Ready (\ref{ch:robo0})
  that will be called after the virtual ADC acquired a new sample.



  The typical way of using this package is as follows:

\begin{itemize}
  \item   The user codes an implementation of the interface Callback\_Handler (\ref{ch:robo12})
    with the processing code (typically) included in the Sample\_Ready (\ref{ch:robo0}) 
    procedure. 
  \item   The user creates a Data\_Source (\ref{ch:robo9}) (see Fakedsp (\ref{ch:robo21}).Data\_Streams (\ref{ch:robo20})) that will 
    be Read (\ref{ch:robo6}) by the virtual ADC in order to get the "virtually 
    acquired" samples.  The easiest way to create a Data\_Source (\ref{ch:robo9}) is to 
    use the "broker constructor" in Fakedsp (\ref{ch:robo21}).Data\_Streams (\ref{ch:robo20}).Files (\ref{ch:robo10}).
  \item   The user creates a Data\_Destination (\ref{ch:robo8}) (see Fakedsp (\ref{ch:robo21}).Data\_Streams (\ref{ch:robo20})) that will 
    be written by the virtual DAC with the value produced by the processing
    code written by the user.  
    The easiest way to create a Data\_Source (\ref{ch:robo9}) is to 
    use the "broker constructor" in Fakedsp (\ref{ch:robo21}).Data\_Streams (\ref{ch:robo20}).Files (\ref{ch:robo10}).
\end{itemize}



\end{document}
