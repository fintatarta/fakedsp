% Document: ./src/fakelib/fakedsp-card.ads
% Source: ./src/fakelib/fakedsp-card.ads
% Generated with ROBODoc Version 4.99.43 (Apr  5 2019)
\documentclass{article}
\usepackage{makeidx}
\usepackage{graphicx}
\oddsidemargin 0.25 in 
\evensidemargin 0.25 in
\marginparwidth 0.75 in
\textwidth 5.875 in
\setlength{\parindent}{0in}
\setlength{\parskip}{.08in}

\pagestyle{headings}
\title{FakeDSP API}
\author{Generated with ROBODoc Version 4.99.43 (Apr  5 2019)
}
\makeindex
\begin{document}
\maketitle
\printindex
\tableofcontents
\newpage

<a class="menuitem" href="./fakedsp-card_adsFakedsp2FFakedsp2Ecard.tex#robo22">Fakedsp.card</a><a class="menuitem" href="./robo_methods.tex#robo_top_of_doc">Methods</a>\subsection{Fakedsp.card/Read\_ADC}
\index{unsorted!Read\_ADC}\index{Methods!Read\_ADC}
\textbf{SOURCE:}\hspace{0.08in}\begin{verbatim}
   function Read_ADC return Sample_Array
     with Pre => Current_State > Sleeping;
   
   function Read_ADC return Sample_Type
     with Pre => Current_State > Sleeping and ADC_DMA_Size = 1;
   
   function Read_ADC return Float
     with Pre => Current_State > Sleeping and ADC_DMA_Size = 1;
\end{verbatim}
\textbf{DESCRIPTION:}\hspace{0.08in}
  Read (\ref{ch:robo6}) the sample (or block of samples) previously
  Read (\ref{ch:robo6}) by the virtual ADC.  Note that if this function is called
  more than once before the Sample\_Ready (\ref{ch:robo0}) method of the handler is
  called, the same value is returned.  Samples that are not Read (\ref{ch:robo6})
  (because, say, the processing is too slow) are lost.



  The three functions differ in their return argument. The functions
  that return single samples (Sample\_Type (\ref{ch:robo24}) or Float) can be called
  only if Start (\ref{ch:robo16}) was called with Out\_Buffer\_Size=1


\end{document}
