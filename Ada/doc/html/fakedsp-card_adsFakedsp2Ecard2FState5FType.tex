% Document: ./src/fakelib/fakedsp-card.ads
% Source: ./src/fakelib/fakedsp-card.ads
% Generated with ROBODoc Version 4.99.43 (Apr  5 2019)
\documentclass{article}
\usepackage{makeidx}
\usepackage{graphicx}
\oddsidemargin 0.25 in 
\evensidemargin 0.25 in
\marginparwidth 0.75 in
\textwidth 5.875 in
\setlength{\parindent}{0in}
\setlength{\parskip}{.08in}

\pagestyle{headings}
\title{FakeDSP API}
\author{Generated with ROBODoc Version 4.99.43 (Apr  5 2019)
}
\makeindex
\begin{document}
\maketitle
\printindex
\tableofcontents
\newpage

<a class="menuitem" href="./fakedsp-card_adsFakedsp2FFakedsp2Ecard.tex#robo22">Fakedsp.card</a><a class="menuitem" href="./robo_types.tex#robo_top_of_doc">Types</a>\subsection{Fakedsp.card/State\_Type}
\index{unsorted!State\_Type}\index{Types!State\_Type}
\textbf{DESCRIPTION:}\hspace{0.08in}
   The "virtual acquisition card" can be in one of 3 states: Sleeping,
   Running, End\_Of\_Data.



   At the beginning, before calling the function Start (\ref{ch:robo16}), it
   is in Sleeping; after Start (\ref{ch:robo16}) it goes Running until it ends the
   data from the source; when out of data, the card goes in End\_Of\_Data.

\textbf{SOURCE:}\hspace{0.08in}\begin{verbatim}
   type State_Type is (Sleeping, Running, End_Of_Data);
\end{verbatim}

\end{document}
