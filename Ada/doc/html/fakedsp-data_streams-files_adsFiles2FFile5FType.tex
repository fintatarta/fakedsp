% Document: ./src/fakelib/fakedsp-data_streams-files.ads
% Source: ./src/fakelib/fakedsp-data_streams-files.ads
% Generated with ROBODoc Version 4.99.43 (Apr  5 2019)
\documentclass{article}
\usepackage{makeidx}
\usepackage{graphicx}
\oddsidemargin 0.25 in 
\evensidemargin 0.25 in
\marginparwidth 0.75 in
\textwidth 5.875 in
\setlength{\parindent}{0in}
\setlength{\parskip}{.08in}

\pagestyle{headings}
\title{FakeDSP API}
\author{Generated with ROBODoc Version 4.99.43 (Apr  5 2019)
}
\makeindex
\begin{document}
\maketitle
\printindex
\tableofcontents
\newpage

<a class="menuitem" href="./fakedsp-data_streams-files_adsData5FStreams2FFiles.tex#robo10">Files</a><a class="menuitem" href="./robo_types.tex#robo_top_of_doc">Types</a>\subsection{Files/File\_Type}
\index{unsorted!File\_Type}\index{Types!File\_Type}
\textbf{SOURCE:}\hspace{0.08in}\begin{verbatim}
   type File_Type is (Wav_File, Text_File, Unknown);
\end{verbatim}
\textbf{DESCRIPTION:}\hspace{0.08in}
   Enumeration type representing the currently recognized format.
   This type is here since we want to allow the user to specify
   the format when the source/destination is open, without
   any guessing on the library side.



   It is a good idea to extend this type when a new format is added.


\end{document}
