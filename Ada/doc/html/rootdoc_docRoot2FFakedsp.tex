% Document: ./src/fakelib/rootdoc.doc
% Source: ./src/fakelib/rootdoc.doc
% Generated with ROBODoc Version 4.99.43 (Apr  5 2019)
\documentclass{article}
\usepackage{makeidx}
\usepackage{graphicx}
\oddsidemargin 0.25 in 
\evensidemargin 0.25 in
\marginparwidth 0.75 in
\textwidth 5.875 in
\setlength{\parindent}{0in}
\setlength{\parskip}{.08in}

\pagestyle{headings}
\title{FakeDSP API}
\author{Generated with ROBODoc Version 4.99.43 (Apr  5 2019)
}
\makeindex
\begin{document}
\maketitle
\printindex
\tableofcontents
\newpage

<a class="menuitem" href="./robo_root.tex#robo_top_of_doc">Documentation root</a>\subsection{Root/Fakedsp}
\index{unsorted!Fakedsp}\index{Documentation root!Fakedsp}
\textbf{WHAT IS THIS?:}\hspace{0.08in}
   This is the Ada version of the Fakedsp library. The library provides
   the equivalent of a virtual card with an ADC and a DAC.  The ADC
   acquires samples at a given sampling frequency and every time 
   a sample is ready a 
   At the same time the DAC updates its output.



   The signal processing is typically carried out inside the virtual
   interrupt handler.  The virtual handler can get the samples 
   produced by the ADC by using the function Read\_ADC (\ref{ch:robo15}) and send the 
   processed data to the DAC by using the procedure Write\_DAC (\ref{ch:robo19}).



   Of course, the ADC and DAC are not real; therefore, the ADC samples
   will be Read (\ref{ch:robo6}) from a file and the DAC samples will be written to
   a file.

\textbf{INSTALLING:}\hspace{0.08in}
    The installation is very simple: just put the content of directory 
    and that's it!  If you use the Adacore project manager

\textbf{HOW DO I USE IT?:}\hspace{0.08in}
  The typical workflow is the following:

\begin{itemize}
  \item   Code the virtual interrupt handler (that, remember, will do the 
    processing) by creating a 
    defined in the package Fakedsp.card (\ref{ch:robo22}).
  \item   Create a Data\_Source (\ref{ch:robo9}) and a Data\_Destination (\ref{ch:robo8}) that will be used, 
    respecitvely, from the ADC and the DAC for sample I/O.  The easiest
    way to do this is to use the opening functions provided by 
    the package Files (\ref{ch:robo10})
  \item   Turn on the virtual card by calling the procedure Start (\ref{ch:robo16}) in 
    Fakedsp.card (\ref{ch:robo22})  The code, behind the scenes,  will get the data
    from the Data\_Source (\ref{ch:robo9}), send results to the Data\_Destination (\ref{ch:robo8}) and call
    your handler
\end{itemize}


\textbf{EXAMPLE:}\hspace{0.08in}
   You can find examples in the directory src/examples.  Currently there
   is only one example, namely an example of processing with a notch filter.
   Maybe more examples will be added in the future. This example can be
   used as a starting point for your own processing code.

\textbf{SEE ALSO:}\hspace{0.08in}
   In the usual spirit of Ada, a detailed (enough) description of 
   the interface of the different packages is embedded directly
   in the spec file (extension

\textbf{NOTES:}\hspace{0.08in}



   Well, you can learn... :-)



   I think that Ada is a very nice and modern language that helps you
   writing very robust software easily and in short time. Debugging times
   are typically 10\% of the debugging time of C. (Note: 10\% of..,
   not 10\% smaller...) I abandoned C since I learned Ada.

   

   A main issue with Ada is that is it is not widely known 
   and there is not much material to learn it.  Nevertheless, there 
   are few very good resources



   learned Ada with this.  Ada 95 is a fairly old version of Ada (now we
   are at Ada 2012) and many new features are missing, but it is a good
   introduction to the language.  If you are new to the language,
   breaking the ground with this can be a good idea.
   Adacore is the
   producer of GNAT, the gcc frontend for Ada. In this site you can find
   an introduction to Ada and an introduction to SPARK (a subset of Ada
   suitable for formal checking, nothing to do with Apache...).  I did
   not use them, but they look nice (and relative to more modern Ada
   versions)
   This is a meta-resource, it is a site with lots of 
   links to learning material, libraries, tools, and so on...
   It is writte in a very readable style and it can be a good reference
   resource at least until you do not get used to... (drum roll)
   It is not possible not to mention the
   (in)famous and
   fundamental
   describing the language, it is written in "computer science
   legalese" and if it was just a bit more obscure it would seem
   encrypted. :-) Seriously, this is the ultimate resource to know
   about Ada, although you need to get used to it.  The annotated
   version and the corresponding "Rationale" (that you can find on the
   Adaic site) often can help since they do not need to be as formal as
   the reference text.
   signal to noise ratio
   and many people wiling to help beginners (although you must be nice
   too... this goes without saying...).  There is also a LinkedIn
   group, but it is less active and more for announcements etc. I know
   that there is also an IRC channel, but I am not an IRC user.

   

   You can download open source IDE, compilers, and so on... from the

   

   Oh, yes, if you come definitively to the dark side and want to meet
   with other Adaists, keep an eye on the
   the
   Bruxells.  Often in the program there is the
   interesting talks and introductions for beginners.  Some past editions:

\begin{itemize}
  \item    2019: href:https://fosdem.org/2019/schedule/track/ada/
  \item    2018: href:https://archive.fosdem.org/2018/schedule/track/ada/ 
  \item    2016: href:https://archive.fosdem.org/2016/schedule/track/ada/ 
\end{itemize}



\end{document}
