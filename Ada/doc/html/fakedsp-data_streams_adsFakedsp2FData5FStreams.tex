% Document: ./src/fakelib/fakedsp-data_streams.ads
% Source: ./src/fakelib/fakedsp-data_streams.ads
% Generated with ROBODoc Version 4.99.43 (Apr  5 2019)
\documentclass{article}
\usepackage{makeidx}
\usepackage{graphicx}
\oddsidemargin 0.25 in 
\evensidemargin 0.25 in
\marginparwidth 0.75 in
\textwidth 5.875 in
\setlength{\parindent}{0in}
\setlength{\parskip}{.08in}

\pagestyle{headings}
\title{FakeDSP API}
\author{Generated with ROBODoc Version 4.99.43 (Apr  5 2019)
}
\makeindex
\begin{document}
\maketitle
\printindex
\tableofcontents
\newpage

<a class="menuitem" href="./rootdoc_docRoot2FFakedsp.tex#robo28">Fakedsp</a><a class="menuitem" href="./robo_packages.tex#robo_top_of_doc">Packages</a>\subsection{Fakedsp/Data\_Streams}
\index{unsorted!Data\_Streams}\index{Packages!Data\_Streams}
\textbf{DESCRIPTION:}\hspace{0.08in}
   Since the virtual card is not... a real one (really?) the samples
   Read (\ref{ch:robo6}) by the ADC need to come from some external source and also
   the samples sento to the virtual DAC need to be written somewehere.



   The most obvious choice for said external source/destinations are
   Files (\ref{ch:robo10}).  However, there are many possible format available such as
   WAV, octave, pure text, and so on...  Moreover, the file could be a
   file on disk or, in a Linux environment, the standard input/output or
   a network connection or...



   In order to not commit ourselves to a specific choice and allowing
   for future expansions, the library introduces two abstract
   interfaces that describe the minimum required to a data source/destination
   and that can be specialized to specific formats. Currenty the library
   provides implementations for WAV format and a text-based format
   compatible with the octave text format.



   The two interfaces defined in this package are:

\begin{itemize}
  \item    Data\_Source (\ref{ch:robo9}) are used to Read (\ref{ch:robo6}) data from (surprised?).
  \item    Data\_Destination (\ref{ch:robo8}) used to Write (\ref{ch:robo3}) data (surprised again, I guess)
\end{itemize}


   Both Data\_Source (\ref{ch:robo9}) and Data\_Destination (\ref{ch:robo8}) have a sampling frequency
   and one or more channels.  They can I/O both values of type
   Sample\_Type (\ref{ch:robo24}) (closer to what actually happen with a real ADC/DAC)
   or of type Float.  We decided of allowing saving Float values
   since in some case we could want to post-process the output produced
   by the processing without the additional noise due to the quantization
   done in order to send data to the DAC.

\textbf{NOTE:}\hspace{0.08in}
   Currently most of the code is designed to work with a single channel
   only.  Maybe this will change in the future.


\end{document}
