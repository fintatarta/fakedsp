% Document: ./src/fakelib/fakedsp-data_streams-files.ads
% Source: ./src/fakelib/fakedsp-data_streams-files.ads
% Generated with ROBODoc Version 4.99.43 (Apr  5 2019)
\documentclass{article}
\usepackage{makeidx}
\usepackage{graphicx}
\oddsidemargin 0.25 in 
\evensidemargin 0.25 in
\marginparwidth 0.75 in
\textwidth 5.875 in
\setlength{\parindent}{0in}
\setlength{\parskip}{.08in}

\pagestyle{headings}
\title{FakeDSP API}
\author{Generated with ROBODoc Version 4.99.43 (Apr  5 2019)
}
\makeindex
\begin{document}
\maketitle
\printindex
\tableofcontents
\newpage

<a class="menuitem" href="./fakedsp-data_streams_adsFakedsp2FData5FStreams.tex#robo20">Data_Streams</a><a class="menuitem" href="./robo_packages.tex#robo_top_of_doc">Packages</a>\subsection{Data\_Streams/Files}
\index{unsorted!Files}\index{Packages!Files}
\textbf{DESCRIPTION:}\hspace{0.08in}
    The user can use as data stream for sample I/O Files with different
    format.  Every implemented format corresponds to a different implementation
    of the Data\_Source (\ref{ch:robo9}) or Data\_Destination (\ref{ch:robo8}) interface defined in the package
    Data\_Streams (\ref{ch:robo20}).



    In order to make it easier to manage different format at runtime,
    this package acts as a kind of "broker" that recognizes the format
    and create the correct object to handle it.



    The interface of this package is really simple: it provides two
    Open functions (Open.Source (\ref{ch:robo27}) and Open.Destination (\ref{ch:robo26})) that look
    at the filename to guess the format if this is not explicitely
    specified by the user.  If given the special filename the opening
    functions use the standard input/output.






    Sometimes it is necessary to give to the procedure opening the
    source/destination some information that cannot be found in the
    file itself.  For example, if the file is just a raw sequence
    of samples, it is not possible to Read (\ref{ch:robo6}) from it the sampling
    frequency.  As another example, when we open a data destination
    that can accept both Sample\_Type (\ref{ch:robo24}) and Float formats, we cannot know
    which format the user desires.



    In order to solve this kind of problems we allow to append to the
    filename an option string.  The option string is separated from
    the filename by "::" and it is a sequence of assignaments
    key=value separated by commas.  For example,






    is a filename that identifies the file



    Note that the option section is defined as the part of the filename
    that goes
    filename






    is a filename that identifies the file
    pretty silly filename, I agree...) with option



    Note that this structure does not allow to have neither "::" nor ","  in an option
    value.  Currently no escape mechanism is used, maybe later.



    Finally, notes that the option mechanism applies
    For example, if we open a data detination with name






    the output samples will be written to the standard output
    in floating point format.


\end{document}
