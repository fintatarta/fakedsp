% Document: ./src/fakelib/fakedsp.ads
% Source: ./src/fakelib/fakedsp.ads
% Generated with ROBODoc Version 4.99.43 (Apr  5 2019)
\documentclass{article}
\usepackage{makeidx}
\usepackage{graphicx}
\oddsidemargin 0.25 in 
\evensidemargin 0.25 in
\marginparwidth 0.75 in
\textwidth 5.875 in
\setlength{\parindent}{0in}
\setlength{\parskip}{.08in}

\pagestyle{headings}
\title{FakeDSP API}
\author{Generated with ROBODoc Version 4.99.43 (Apr  5 2019)
}
\makeindex
\begin{document}
\maketitle
\printindex
\tableofcontents
\newpage

<a class="menuitem" href="./rootdoc_docRoot2FFakedsp.tex#robo28">Fakedsp</a><a class="menuitem" href="./robo_packages.tex#robo_top_of_doc">Packages</a>\subsection{Fakedsp/Fakedsp}
\index{unsorted!Fakedsp}\index{Packages!Fakedsp}
\textbf{DESCRIPTION:}\hspace{0.08in}
  The library Fakedsp allows  to Write (\ref{ch:robo3}) code with a DSP-style on a normal PC.
  I wrote this to help my students in doing their lab activities.



  The user of the library will be most probably interested in
  the package Fakedsp.card (\ref{ch:robo22}) that provides the interface to the
  "virtual DSP card."



  Another package of interest is probably Fakedsp.Data\_Streams (\ref{ch:robo20}).Files (\ref{ch:robo10})
  that implement the Data\_Source (\ref{ch:robo9})/Data\_Destination (\ref{ch:robo8}) interfaces
  (defined in Fakedsp.Data\_Streams (\ref{ch:robo20})) that represent the main abstraction
  for data I/O.


\end{document}
